\documentclass[11pt,a4paper]{article}

\usepackage[utf8]{inputenc}
\usepackage[english]{babel}
\usepackage[T1]{fontenc}

\usepackage{amsmath,amssymb,amsfonts}

\title{Advanced Language Processing - Assignment 1}
\author{Sebastian Paaske Tørholm}

\usepackage{hyperref}
% Format function name and link to it on Hoogle
\newcommand{\hoogle}[1]{\href{http://www.haskell.org/hoogle/?hoogle=#1}{\texttt{#1}}}

\begin{document}
\maketitle

\section{Implementation}

I chose to do my implementation in Haskell, primarily due to easy access to
parser combinators and ADTs.

\subsection{Syntax tree}

For storing the syntax tree, I make use of Haskell's excellent ADTs. The ADTs
follow the grammar rather closely. I have chosen to group together all the
assignment statements to non-memory into one instruction, with the right hand
side being an "expression", capturing the same cases shown in the grammar. This
choice seemed natural, and made specifying kill sets quite beautiful.

\subsection{Parser}

For parsing, I use \hoogle{ReadP}. The parser is a fairly straight forward
parser combinator based parser. More or less every rule from the grammar is
given its own subparser, so it should be easy to follow. I've chosen to allow
arbitrary whitespace between tokens, to provide some freedom for formatting.
Additionally, I've chosen to let instructions not be separated by commas, as
the commas are unnecessary and don't provide a benefit to readability.

The \texttt{Parser} module provides two functions: 

\begin{description}
    \item[\texttt{parseFile}] Parses the contents of a file into a syntax tree.
    \item[\texttt{parseString}] Parses the contents of a string into a syntax tree.
\end{description}

Any parse errors lead to an undescribed failure result, due to the nature of
\hoogle{ReadP}.

The parser comes with a unit test, which assures that it parses the fibonacci
example correctly.

\subsection{Liveness analysis}

Liveness analysis is performed in the module \texttt{Analysis}. The
function \texttt{liveness} is called on a program, giving a result of type
\texttt{Liveness}.

A \texttt{Liveness} value contains a list of tuples, each tuple having
two parts: The function name, and a list of tuples, each containing an
instruction, its in-set, and its out-set.

\texttt{livenessFun} is called on each function of the program, and does
precalculation of the successor, gen and kill sets, as well as where each
label is in the program. This is then fed to \texttt{livenessIter}, which
performs fixed-point iteration on the in- and out-sets.

Since gen and kill need to be referenced across different lines, arrays are
used for constant-time access. Sets are stored as sets, giving us the set
operations we need.

The \texttt{Analysis} module provides two functions:

\begin{description}
    \item[\texttt{liveness}] Runs liveness analysis on a given program.
    \item[\texttt{livenessFile}] Parses the contents of a file, and runs liveness analysis on it.
\end{description}

Manual testing has been done, and the output looks sane.

\section{Example programs}
Three example programs are attached in the \texttt{programs} folder:

\begin{description}
    \item[\texttt{fib.int}] The Fibonacci function (Figure 3.3)
    \item[\texttt{gcd.int}] The GCD function (Exercise 3.1)
    \item[\texttt{easter.int}] A function for calculating when it'll be easter on a given year.
\end{description}

\section{Results of liveness analyses}

The results of running the liveness analysis on the example programs can be
found in the \texttt{results} folder.

\end{document}

